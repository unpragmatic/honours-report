\chapter{Introduction} \label{C:intro}
One of the principal concerns of programming language designers
and implementors is speed. This paper addresses those concerns
in the context of Kihi: a minimal functional, compositive,
and concatenative language. 

These qualities make it an 
interesting subject for optimisation, in particular, the large
valley of abstraction between the program semantics and bare
metal brings about many challenges and opportunities.




\section{Kihi}
NOTE: maybe move this to introduction.
The Kihi programming language consists of only six types of terms.
Furthermore, it can be considered a functional, concatatenative,
and compositional language. Culmulatively, these features make it
an interesting subject for researching. An explanation of these
terms is provided in the upcoming sections. 