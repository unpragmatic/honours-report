\chapter{Introduction} \label{C:intro}
One of the principal concerns of programming language designers
and implementers is performance. This project addresses those concerns in the context of Kihi: a minimal functional, compositional, and concatenative language. Specifically, this project is the creation of a performance focused implementation of the Kihi language. This report outlines the various optimisation methods used in this implementation and provides an evaluation of the results. However firstly the report begins by providing the necessary background information on the Kihi language and the various qualities that make it an interesting optimisation subject. 


\todo[inline]{
In addition, the operational semantics of the language do not
resemble actual bare metal instructions.

These qualities make it an 
interesting subject for optimisation, in particular, the large
valley of abstraction between the program semantics and bare
metal brings about many challenges and opportunities.
}


\todo[inline]{
NOTE: maybe move this to introduction.
The Kihi programming language consists of only six types of terms.
Furthermore, it can be considered a functional, concatatenative,
and compositional language. Culmulatively, these features make it
an interesting subject for researching. An explanation of these
terms is provided in the upcoming sections. 

}