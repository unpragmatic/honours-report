\chapter{Introduction} \label{C:intro}
One of the principal concerns of programming language designers
and implementers is speed. This paper addresses those concerns
in the context of Kihi: a minimal functional, compositional,
and concatenative language. Specifically, this paper outlines
an performance focused implementation of the Kihi language in
Rust. The aforementioned properties of Kihi make it an interesting
optimisation subject. They are rarely found in mainstream languages,
although recently functional concepts have become much more widespread.
The details of these properties are provided in the background
section of the report.


\todo[inline]{
In addition, the operational semantics of the language do not
resemble actual bare metal instructions.

These qualities make it an 
interesting subject for optimisation, in particular, the large
valley of abstraction between the program semantics and bare
metal brings about many challenges and opportunities.
}


\todo[inline]{
NOTE: maybe move this to introduction.
The Kihi programming language consists of only six types of terms.
Furthermore, it can be considered a functional, concatatenative,
and compositional language. Culmulatively, these features make it
an interesting subject for researching. An explanation of these
terms is provided in the upcoming sections. 

}