\chapter{Introduction} \label{C:intro}
One of the principal concerns of programming language designers
and implementers is performance. The goal of this project is to address those concerns in the context of Kihi, a programming language created by Timothy Jones and Michael Homer \cite{jones2018practice}. The project's goal was to develop a performance focused implementation of the Kihi language, in particular focused on minimising the time between source code input and execution completion. This report describes the performance focused design underlying the implementation, also known as the Kihi Runner. This also includes an explanation of a potentially novel optimisation technique used in this project.


This report begins by providing an introduction to the Kihi language in chapter \ref{C:background}. This includes an explanation of the language and its origins and also a brief explanation of surrounding material such as related programming paradigms. Readers familiar with this material may skip to chapter \ref{C:implementation} which discussed the heart of the project: the implementation. This includes details of the implementation's design and the optimisation techniques used to improve performance. Most interestingly, section \ref{sec:implementation_optimisation} introduces the aforementioned optimisation technique capable of dynamically finding more efficient representations of programs.

\todo[inline]{
    Move to background:
In addition, the operational semantics of the language do not
resemble actual bare metal instructions.

These qualities make it an 
interesting subject for optimisation, in particular, the large
valley of abstraction between the program semantics and bare
metal brings about many challenges and opportunities.
}

