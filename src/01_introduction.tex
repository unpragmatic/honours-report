\chapter{Introduction} \label{C:intro}
A principal concern of many programming languages is performance. The goal of this project is to address those concerns in the context of the Kihi programming language. Specifically, this project attempts to address those concerns by creating a performance focussed implementation of the language. In the context of this project, performance is determined by the time between source code and completed execution. This implementation, also referred to as the Kihi Runner, is distinguished by its performance focussed features, in particular, an optimisation pipeline capable of dynamically constructing more efficient and compact representations of Kihi programs. How effective these optimisations are is the primary question this report attempts to answer.


Firstly however, the report begins with a treatment of surrounding background material in chapter \ref{C:background}. This most importantly includes a explanation of the Kihi language and its origins, with special attention paid to its raison d'être and unique characteristics. However, this material may be skipped by readers who are already familiar with the language. The follow chapter presents the thrust of this project's work. It describes the design and implementation of the Kihi Runner, illustrating its various features. Most interestingly, this includes an explanation of the aforementioned optimisation pipeline. 

The report closes with an evaluation of the Kihi Runner, focussing on whether the optimisation pipeline has been effective. This evaluation examines the performance of the Kihi Runner on a variety of micro benchmarks, which look at the performance of individual components, and macro benchmarks, which look at the performance of the entire system.