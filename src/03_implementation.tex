\chapter{Design and Implementation} \label{C:implementation} 
This chapter discusses the details of the Kihi implementation presented in this report. Specifically, this section shows how the semantics outlined in chapter \ref{C:background} can be implemented and discusses various performance driven design decisions, such as language choice and optimisation strategies. But firstly, in order to guide the discussion, an overview of the actual program is presented below.

\section{Overview}
The executable described within this report is capable of executing Kihi source directly. The executable takes as an argument the path to a text file contain Kihi source code, encoded in either UTF-8 or ASCII, and subsequently runs the program, emitting any outputs to standard output. Internally,the source code is first parsed into an intermediary representation before execution. This     intermediary representation is mostly identical to the grammar shown in figure \ref{fig:kihi_grammar} barring implementation details, debugging symbols, and optimisation mechanisms. The optimisation mechanisms are of particular interest and are discussed in greater detail in section \ref{sec:implementation_optimisation}. The execution process involves selecting an executor, which can be understood as virtual machines, and utilising it to execute the intermediary representation. Test

The executable also has various configuration options avaliable at compile time and run time. At compile time feature flags can be set to include executors in the binary. By default the binary is only built with the reduce executor, which utilises term rewriting operational semantics, however an executor based on stack operational semantics can be included in addition to a previous implementation of the reduce executor.

During compilation, options such as profiling, printing to standard out, and 

\section{Operational Semantics to Code}
A basic implementation of Kihi is straight forward. An almost direct translation of the operational semantics presented in figure \ref{fig:term_rewriting_op_sem} and figure \ref{fig:stack_op_sem} is possible as shown by the pseudocode in figure \ref{fig:term_rewriting_pseudocode} and figure \ref{fig:stack_pseudocode}. Furthermore, the pseudocode in the figures are an accurate representation of the respective implementations avaliable in the actual binary.



this was the approach taken for the initial implementations. The appendix provides commit hashes for particularly noteworthy implementations, however the latest commit provides backwards compatibility to run later editions for benchmarking and prosperity's sake.

The language

\todo[inline]{wordy wtf change this.}



\section{Language Choice}

\section{Optimisation}\label{sec:implementation_optimisation}
There were two primary optimisations implemented in the Kihi virtual
machine. Firstly, symbolic analysis: finding commonly occuring or
meaningful sequences of terms, and secondly symbol optimisation: the
process of dynamically constructing a more efficient method of executing
a symbol. 

\subsection{Symbolic Analysis}
\todo[inline]{Find reference probably simmilar idea exists in lit}
The Kihi virtual machine implements a relatively simple symbol analyser.
The base symbolic analyser combines adjacent operators into symbols. The
motivation behind this method

\section{Execution Style}


\section{Benchmark Suite}

\begin{figure}[htb]
    \centering
    \begin{lstlisting}
execute_program(input: String):
    terms: []Term := parse_program(input)
    
    reductible_term: Index := find_reductible_term(terms)
    while (reductible_term != -1) {
        reduce_term(terms, reductible_term)
    }

find_reductible_term(terms: []Term) -> Index:
    for i: Index in 0..|terms| {
        if terms[i] is 'apply'
            and terms[i+1] is abstraction => return i
        else if terms[i] is 'left'
            and terms[i+1] and terms[i+2] are abstractions => return i
        else if terms[i] is 'drop'
            and terms[i+1] => return i
        ... and so on for each inference rule
        }
    }
    return -1

reduce_term(terms: []Term, term: Index):
    if terms[index] is 'apply'
        terms[index..index+1] = terms[index+1]
    else if terms[index] is 'left'
        terms[index..index+2] = [terms[index+2], ...terms[index+1]]
    else if terms[index] is 'drop'
        terms[index..index+1] = []
    ... and so on for each inference rule
    \end{lstlisting}
    \caption{Pseudocode for a term rewriting based Kihi implementation}
    \label{fig:term_rewriting_pseudocode}
\end{figure}
\begin{figure}[htb]
    \centering
    \begin{lstlisting}
execute_program(input: String):
    program: []Term := parse_program(input)
    
    stack: []Term := []
    
    while(|terms| != 0) {
        term := terms.pop()
        if term is abstraction => stack.push(term)
        else if term is 'apply' => program.append( stack.pop() )
        else if term is 'left' => {
            arg1 = stack.pop()
            arg2 = stack.pop()
            stack.push( [arg2] ++ arg2 )
        }
        ... and so on for each inference rule
    }
    \end{lstlisting}
    \caption{Pseudocode for a stack based Kihi execution}
    \label{fig:stack_pseudocode}
\end{figure}