\chapter{Background} \label{C:background} 

\section{Kihi}
The Kihi programming language consists of only six types of terms. Either
one of five operators or an abstraction: a sequence of Kihi terms
captured by parenthesis. 

The complete grammar for the language is given
in figure \ref{fig:grammar}. The semantics of the language make it
interesting from a research perspective. The limited expressiveness of
the language prevents the common approach to optimisation where expressive
abstractions can be compiled to efficient machine code. [cite static
dispatch optimisation]. Furthermore, the abstractions/semantics of Kihi are
also far disjoined from bare metal. This incurs the additional overhead of
mapping Kihi to machine instructions.

It limits the space of possible
optimisations and is also far detracted from actual bare metal instructions.


\begin{figure}[htb]       
    \centering
    \begin{grammar}
    <program> ::= { <term> }

    <abstraction> ::= '(' <term> ')'

    <term> ::= <abstraction>
        \alt 'apply'
        \alt 'left'
        \alt 'right'
        \alt 'copy'
        \alt 'drop'
    
      \end{grammar}
    \caption{}
    \label{fig:grammar}
\end{figure}



\begin{figure}
    \centering
    \begin{enumerate}
        \item \textbf{Apply} releases the sequence of terms captured by an abstraction.
        \item \textbf{Left} places the second abstraction at the start of the first
        \item \textbf{Right} places the second abstraction at the end of the first
        \item \textbf{Copy} copies an abstraction
        \item \textbf{Drop} deletes an abstraction
    \end{enumerate}
    \caption{English explanation of operators}
    \label{fig:operator explanation}
\end{figure}

Of these terms, only one acts as a value

\section{Execution Style}
There are two main execution styles that can be used to execute
a Kihi program: a term rewriting approach and a stack based 
approach. These two approachs are identical for side-effect
free and terminating programs, however, the observed
behaviour for other varieties of programs can differ
substantially depending on the execution style. The rest of 
this section describes these two execution styles and 
concludes by contrasting their behaviour.

\subsubsection{Term Rewriting}
Term rewriting is the process of finding
and replacing reducible terms with their reduced form. For Kihi,
this means finding terms which are directly followed by
some number of abstractions. The specific number of abstractions
required is determined by the term and can be thought of as the
number of arguments. This execution paradigm allows execution
to occur anywhere in the program because a reducible term may
be found anywhere in the program and this reduction can be
thought of as an execution step.

The operational semantics for a term rewriting based execution of
Kihi is shown is figure \ref{fig:term_rewriting_op_sem}.

\begin{figure}[p]
    \centering
    \begin{frame}{}
    \infax{
        Program = \langle t_0 : t_1 : \dots : t_n \rangle \ |\ t \in Terms
    }

    \infax{
        Abstractions = \{ (p) \mid p \in Programs \}
    }

    \infax{
        Terms = \{\mbox{apply}, \mbox{left}, \mbox{right}, \mbox{copy}, \mbox{drop} \}
        \cup Abstractions \\
    }

    \infrule[Apply]
        {t \in Abstractions \andalso t = (p)}
        {\kihi{ \mbox{apply} : t } \leadsto p}

    \infrule[Left]
        {t_1, t_2 \in Abstractions \andalso t_1 = (p_1)}
        {
            \kihi{ \mbox{left} : t_1 : t_2 }
            \leadsto 
            \kihi{ t_2 : p1 }
        }

    \infrule[Right]
        {t_1, t_2 \in Abstractions \andalso t_1 = (p_1) \andalso t_2 = (p_2)}
        {
            \kihi{ \mbox{right} : t_1 : t_2 }
            \leadsto 
            \kihi{ p_1 : t_2 }
        }

    \infrule[Copy]
        {t \in Abstractions}
        {\kihi{ \mbox{copy} : t } \leadsto \kihi{ t : t }}

    \infrule[Drop]
        {t_2 = (p_1)}
        {\langle \mbox{drop} : t_2 \rangle \leadsto \emptyset}

    \infrule[Term Rewriting]
        {p_1 \leadsto p_2}
        {\langle p : p_1 : p' \rangle \leadsto p : p_2 : p'}
    \end{frame}
    \caption{Term rewriting operational semantics for Kihi}
    \label{fig:term_rewriting_op_sem}
\end{figure}
\begin{figure}[p]
    \centering
    \begin{frame}{}
    \infax{
        Program = \langle t_0 : t_1 : \dots : t_n \rangle \ |\ t \in Terms
    }

    \infax{
        Stack = \langle v_0 : v_1 : \dots : v_n \rangle \ |\ v \in Abstractions
    }

    \infax{
        Abstractions = \{ (p) \mid p \in Programs \}
    }

    \infax{
        Terms = \{\mbox{apply}, \mbox{left}, \mbox{right}, \mbox{copy}, \mbox{drop} \}
        \cup Abstractions \\
    }

    \infrule[Abstraction]
        {t \in Abstractions}
        {\kihi{ \dots : v_n}, \kihi{ t } \leadsto \kihi{ \dots : v_n : t}, \kihi{ }}

    \infrule[Apply]
        {}
        {\kihi{ \dots : v_n}, \kihi{ `apply' } \leadsto \kihi{ \dots }, \kihi{ v_n }}

    \infrule[Left]
        {v_1 = (p)}
        {\kihi{ \dots : v_2 : v_1}, \kihi{ `left' } \leadsto \kihi{ \dots : (v_2 : p) }, \kihi{ }}

    \infrule[Right]
        {v_1 = (p)}
        {\kihi{ \dots : v_2 : v_1}, \kihi{ `left' } \leadsto \kihi{ \dots : (p : v_2) }, \kihi{ }}

    \infrule[Copy]
        {}
        {\kihi{ \dots : v_n}, \kihi{ `copy' } \leadsto \kihi{ \dots : v_n : v_n }, \kihi{ }}

    \infrule[Drop]
        {}
        {\kihi{ \dots : v_n}, \kihi{ `drop' } \leadsto \kihi{ \dots }, \kihi{ }}

    \infrule[Stack Machine]
        {s, p \leadsto s', p'}
        {s, \langle \dots : p \rangle \leadsto s', \kihi{ \dots : p'}}
    \end{frame}
    \caption{Stack operational semantics for Kihi}
    \label{fig:stack_op_sem}
\end{figure}


\subsubsection{Stack Based}
Stack based execution is a model of execution where each term
has a specfic effect on a global stack. The terms are typically
either processed in a right to left order or vice versa. This
restricts execution to the rightmost or leftmost term. Furthermore,
the use of stack restricts the arguments accessible to each term.
Stack based execution imagines the terms of the program as
instructions for a stack machine. The terms are scanned
right to left and an action performed depending on the term
encountered. For instance, whenever an abstraction is found
it is pushed onto the stack and whenever a drop is found the
topmost value of the stack is popped and discarded.

\subsubsection{Differences}
The most fundamental difference these methods of execution is
revealed when term rewriting is broken down into two fundamental steps.

1. Seek
    - Find the term to execute
2. Execution
    - Execute that term

In contrast, stack based methods do not involve a seek phase. This
can significantly improve performance, but also shows that stack
based systems are a subset of term rewriting based systems. In fact,
a term rewriting based system seeking from right to left is usually 
equivilent to a stack based system when the program. The edgecase
being when the program being executed contains barriers. In this case,
term rewriting based systems are able to find executable portions
beyond the barrier.

A term rewriting based virtual machine will be able to execute
all programs a stack based machine is able to. The proof of this
is relatively simple. One can imagine a term rewriting machine 
that searches for reducible terms right to left as equivilent to
a stack machine for terminating programs.



A program representing an infinite data structure, such as the count
program which outputs a continous stream of numbers, cannot be
processed through a normal stack based approach. There may be alterations
to the stack based approach which may allow values to be pushed 
through to the left hand side of the program.


an abstraction is
encountered it is pushed onto the stack and when a term (reword)
is found the arguments are popped off the stack and the 
result, if it is an abstraction, is pushed onto the stack.

The first approach involves finding and replacing
reducible terms with their reduced form. The essence of this
execution style is shown in figure 
\ref{fig:term_rewriting_pseudocode} which provides a pseudocode
description of the essential algorithm.

These two different approaches are identical for
terminating programs but can display very different behaviour
for non-terminating programs.
Any acknowledgments should go 
in here, between the title page and the table of contents.  The 
acknowledgments do not form a proper chapter, and so don't get a 
number or appear in the table of contents.
