\chapter{Background} \label{C:background}
This chapter aims to provide an understanding of the surrounding background material. In particular, this chapter provides a explanation of the variety of concepts surrounding the Kihi programming language alongside an explanation of the language itself. This includes an introduction to the programming paradigms that inspired Kihi such as functional programming, concatenative programming, and compositional programming. This is followed by a thorough explanation of Kihi, examining topics such as its purpose, grammar, and evaluation methods.

\section{Imperative Programming}
The world of programming languages began with imperative programming languages. These languages described a program as a series of statements, each representing an operation for the computer to perform. For low level languages such as machine code and assembly, these statements can be an almost direct reflection of the actual computer's behaviour. However, writing programs in these languages is quickly becomes impractical as the complexity of the application grows. Higher level languages, such as Java and C++, overcome this by providing higher level abstractions, such as objects and procedures. These not only provide a mechanism for organisation and managing complexity, but are also capable of expressing in a single statement what might have taken hundreds in a lower level language.
[citation needed?]


\section{Functional Programming}
\begin{figure}[htb]
    \centering
    \begin{lstlisting}[language=Java]
public class Main {
    public static int imperative_sum(List<Integer> xs) {
        int sum = 0;
        for (int x : xs) {
            sum += x;
        }
        return sum;
    }

    public static int functional_sum(List<Integer> xs) {
        if (xs.isEmpty()) {
            return 0;
        }

        return xs[0] + functional_sum(xs.sublist(1, xs.size()));
    }
}
\end{lstlisting}
    \caption{A imperative implementation of summation and a functional implementation of summation.}
    \label{fig:imperative_vs_functional_example}
\end{figure}

Standing in contrast to imperative programming is functional programming. In contrast to imperative programming languages which describe a program using a series of statements, functional programming describes a program as a series of function evaluations. A function represents a transformation. A function is called pure when the transformation has no side effects, meaning the function has no observable effects aside from emitting the output \cite{FunctionalProgrammingHaskellWiki2019}. In contrast, statements in imperative languages are often impure.


In order to be considered a functional language, the language must have first-class functions, which are functions that can be treated as values. Java 8+ provides this in the form of lambda expressions, allowing for programs to written in a functional manner. An example of this is shown in figure \ref{fig:imperative_vs_functional_example}. Colloquially however, Java and languages like it are not considered functional programming languages since the predominant paradigm is not functional programming, rather it is object orientation or some other paradigm respectively.

Functional languages often distinguish between 
\todo[inline]{Address purity, reference haskell book}
\todo[inline]{Address difficult of optimising a pure language}


\section{Concatenative and compositional languages} 
Compositional and concatenative are two labels that are often used in tandem when describing a programming language. The former describes a language where the main interaction is function composition as opposed to function application. The latter is used to describe a language where the source code of two programs can be meaningfully concatenated together. Figure \ref{fig:applicative_vs_compositional_example} demonstrates the differences between applicative and compositional languages. These two labels are often found together because concatenative languages naturally encourage a compositional approach to programming since it allows for programs to understood as a pipeline. In contrast, an applicative concatenative language has no simple mental model.

The first label is used to describe a programming languages where the source code of two programs can be concatenated together into a meaningful program, and the second a language where the juxtaposition of terms denotes function composition. These labels are often found together because these properties are closely associated. Any language that is compositional can easy express pipeline processes by simply enumerating the stages. It is then also clear that the source code of such programs can also be concatenated to form a larger pipeline with the combined stages of the two.

\begin{figure}[!htb]
    \centering
    \begin{lstlisting}
: ONE-VALUE-REMAINING DEPTH 1 = ;

: SUM BEGIN + ONE-VALUE-REMAINING UNTIL ;
\end{lstlisting}
    \caption{A Forth program that sums all the elements on the stack}
    \label{fig:forth_example}
\end{figure}

\begin{figure}[htb]
    \centering
    \begin{lstlisting}
Applicative:
    f g x => (f(g))(x) 

Compositional:
    f g x => f(g(x))
\end{lstlisting}
    \caption{A demonstration of the differences between an applicative and compositional language.}
    \label{fig:applicative_vs_compositional_example}
\end{figure}



\begin{figure}
    \renewcommand{\labelitemi}{$\textendash$}
    \begin{enumerate}
        \item \textbf{Apply} releases the sequence of terms captured by an abstraction:
        \begin{itemize}
            \item \lstinline{apply (x)} $\leadsto$ \lstinline{x}
        \end{itemize}

        \item \textbf{Left} places the second abstraction at the start of the first:
        \begin{itemize}
            \item \lstinline{left (x) (y)} $\leadsto$ \lstinline{((y) x)}
        \end{itemize}

        \item \textbf{Right} places the second abstraction at the end of the first:
        \begin{itemize}
            \item \lstinline{right (x) (y)} $\leadsto$ \lstinline{(x (y))}
        \end{itemize}

        \item \textbf{Copy} copies an abstraction:
        \begin{itemize}
            \item \lstinline{copy (x)} $\leadsto$ \lstinline{(x) (x)}
        \end{itemize}

        \item \textbf{Drop} deletes an abstraction:
        \begin{itemize}
            \item \lstinline{drop (x)} $\leadsto$ \lstinline{}
        \end{itemize}
    \end{enumerate}
    \caption{An explanation of the five Kihi operators}
    \label{fig:operator_explanation}
\end{figure}

\section{Kihi}
The Kihi programming language is the focus of this report so it is worthwhile to explain its structure and operation in detail. This section begins with an explanation of the language's grammar and follows up with an exploration of two ways a Kihi program can be executed.

\subsection{Grammar}
As previous mentioned, Kihi consists of only six types of terms: one value type and five operators that can manipulate values. The value type, more commonly referred to as an abstraction, is a Kihi program captured by parenthesis. The is illustrated by figure \ref{fig:kihi_grammar} which shows the formal grammar of Kihi and figure \ref{fig:kihi_example} which provides an example of a Kihi program. The operators are plainly explained in figure \ref{fig:operator_explanation}.

\begin{figure}[htb]
    \begin{mdframed}
    
    \begin{grammar}
    <program> ::= { <term> }

    <abstraction> ::= '(' <term> ')'

    <operator> ::= 'apply' | 'left' | 'right' | 'copy' | 'drop'

    <term> ::= <abstraction> | <operator>
    \end{grammar}
    \end{mdframed}
    \caption{The grammar of Kihi}
    \label{fig:kihi_grammar}
\end{figure}

\begin{figure}[htb]
    \centering
    \begin{lstlisting}
apply right copy right (apply apply left (right right (apply) copy)) (apply left (apply apply left left ()) copy right (apply apply left (apply) apply left left () apply left (copy)) apply left left ()) (drop)
\end{lstlisting}
    \caption{A Kihi program that counts from zero to infinity}
    \label{fig:kihi_example}
\end{figure}


\todo[inline]{Add citation for the counting program?}
\todo[inline]{Provide examples of how the operators work}
\todo[inline]{The exact behaviour of the operators are described later}


\subsection{Evaluation Methods}\label{sec:background_evaluation_methods}
There are two main execution styles that can be used to execute Kihi programs: a term rewriting based approach and a stack based approach. These two approaches are identical in terms of output for terminating programs, however performance and output can differ significantly for certain classes of programs. This section describes the two execution styles and explains when they differ.

\subsubsection{Term Rewriting}
Term rewriting is the process of finding
and replacing reducible terms with their reduced form. For Kihi,
this means finding terms which are directly followed by
some number of abstractions. The specific number of abstractions
required is determined by the term and can be thought of as the
number of arguments. This execution paradigm allows execution
to occur anywhere in the program because a reducible term may
be found anywhere in the program and each reduction can be
thought of as an execution step. The operational semantics for a term rewriting based execution of
Kihi is shown in figure \ref{fig:term_rewriting_op_sem}. Informally, term rewriting can be understood by referring to figure \ref{fig:operator_explanation}. The process is simply consists of finding a term where the rules shown in figure \ref{fig:operator_explanation} can be applied and applying them.

% \todo[inline]{reword informal}
\subsubsection{Stack}
Stack based execution is a model of execution where each term has a specific effect on a stack. The terms are typically either processed in a right to left order or vice versa. In contrast to term rewriting, execution is restricted to a specific next term. In the context of Kihi, the next term is the rightmost term of the program. The operational semantics for a stack based Kihi implementation are provided in figure \ref{fig:stack_op_sem}. Informally, it can be understood as a processes each term from right to left in order where the term's can manipulate the stack and program. When an abstraction is encountered it is simply pushed onto the stack, and when an operator is encountered the arguments it requires are popped off the stack and the result either pushed onto the stack, in the case of left, right, copy, or discard in the case of drop, or appended to the program in the case of apply.

\todo[inline]{reword informal explanation}

\todo[inline]{talk about church numeral encoding}

% The key difference between these methods of execution is
% revealed when term rewriting is broken down into its two fundamental steps. Namely, first finding an executable term, and secondly executing it. The stack based style can be viewed as a simplification of term rewriting where the first step is irrelevant.

% In fact, term rewriting can very easily simulate a stack machine by storing the stack within the program.

\input{figures/04_implementation/term_rewriting_op_sem.tex}
\begin{figure}[htb]
    \centering
    \begin{frame}{}
    \infax{
        Program = \langle t_n \mid t \in Terms \rangle
    }

    \infax{
        Stack = \langle t_n | t \in Abstractions \rangle
    }

    \infax{
        Abstractions = \{ (p) \mid p \in Programs \}
    }

    \infax{
        Terms = \{\mbox{apply}, \mbox{left}, \mbox{right}, \mbox{copy}, \mbox{drop} \}
        \cup Abstractions \\
    }

    \infrule[Apply]
        {t \in Abstractions \andalso t = (p)}
        {\kihi{ \mbox{apply} : t } \leadsto p}

    \infrule[Left]
        {t_1, t_2 \in Abstractions \andalso t_1 = (p_1)}
        {
            \kihi{ \mbox{left} : t_1 : t_2 }
            \leadsto 
            \kihi{ t_2 : p1 }
        }

    \infrule[Right]
        {t_1, t_2 \in Abstractions \andalso t_1 = (p_1) \andalso t_2 = (p_2)}
        {
            \kihi{ \mbox{right} : t_1 : t_2 }
            \leadsto 
            \kihi{ p_1 : t_2 }
        }

    \infrule[Copy]
        {t \in Abstractions}
        {\kihi{ \mbox{copy} : t } \leadsto \kihi{ t : t }}

    \infrule[Drop]
        {t_2 = (p_1)}
        {\langle \mbox{drop} : t_2 \rangle \leadsto \emptyset}

    \infrule[Term Rewriting]
        {p_1 \leadsto p_2}
        {\langle p : p_1 : p' \rangle \leadsto p : p_2 : p'}
    \end{frame}
    \caption{Term rewriting operational semantics for Kihi}
    \label{fig:term_rewriting_op_sem}
\end{figure}

\subsubsection{Term Rewriting vs Stack}
\todo[inline]{
The most fundamental difference these methods of execution is
revealed when term rewriting is broken down into two fundamental steps.

1. Seek
    - Find the term to execute
2. Execution
    - Execute that term

In contrast, stack based methods do not involve a seek phase. This
can significantly improve performance, but also shows that stack
based systems are a subset of term rewriting based systems. In fact,
a term rewriting based system seeking from right to left is usually 
equivilent to a stack based system when the program. The edgecase
being when the program being executed contains barriers. In this case,
term rewriting based systems are able to find executable portions
beyond the barrier.

A term rewriting based virtual machine will be able to execute
all programs a stack based machine is able to. The proof of this
is relatively simple. One can imagine a term rewriting machine 
that searches for reducible terms right to left as equivilent to
a stack machine for terminating programs.



A program representing an infinite data structure, such as the count
program which outputs a continous stream of numbers, cannot be
processed through a normal stack based approach. There may be alterations
to the stack based approach which may allow values to be pushed 
through to the left hand side of the program.


an abstraction is
encountered it is pushed onto the stack and when a term (reword)
is found the arguments are popped off the stack and the 
result, if it is an abstraction, is pushed onto the stack.

The first approach involves finding and replacing
reducible terms with their reduced form. The essence of this
execution style is shown in figure 
\ref{fig:term_rewriting_pseudocode} which provides a pseudocode
description of the essential algorithm.

These two different approaches are identical for
terminating programs but can display very different behaviour
for non-terminating programs.
Any acknowledgments should go 
in here, between the title page and the table of contents.  The 
acknowledgments do not form a proper chapter, and so don't get a 
number or appear in the table of contents.
}
\todo[inline]{Address the addition of side-effectful operators}.