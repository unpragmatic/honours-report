\begin{figure}
    \renewcommand{\labelitemi}{$\textendash$}
    \begin{enumerate}
        \item \textbf{Apply} releases the sequence of terms captured by an abstraction:
        \begin{itemize}
            \item \lstinline{apply (x)} $\leadsto$ \lstinline{x}
        \end{itemize}

        \item \textbf{Left} places the second abstraction at the start of the first:
        \begin{itemize}
            \item \lstinline{left (x) (y)} $\leadsto$ \lstinline{((y) x)}
        \end{itemize}

        \item \textbf{Right} places the second abstraction at the end of the first:
        \begin{itemize}
            \item \lstinline{right (x) (y)} $\leadsto$ \lstinline{(x (y))}
        \end{itemize}

        \item \textbf{Copy} copies an abstraction:
        \begin{itemize}
            \item \lstinline{copy (x)} $\leadsto$ \lstinline{(x) (x)}
        \end{itemize}

        \item \textbf{Drop} deletes an abstraction:
        \begin{itemize}
            \item \lstinline{drop (x)} $\leadsto$ \lstinline{}
        \end{itemize}
    \end{enumerate}
    \caption{An explanation of the five Kihi operators}
    \label{fig:operator_explanation}
\end{figure}